In this section, we review classes of graphs of bounded width.
In particular, we are interested in such cases where the width parameters are exponentially smaller than theoretically possible, as this would allow to solve \#SAT even more efficiently.

% trees
\textbf{Trees.}
Recall that a tree is a connected graph without cycles.
We begin with a fairly obvious observation:

\begin{fact}
	%\textbf{Fact 1.}
	If $G$ is a tree, then \textbf{tw}$(G) = 1$. 
	\label{gw:tree}
\end{fact}

\noindent
This is due to the fact that any tree can be decomposed into bags of size 2.
{\color{lightgray} As there are many ways to construct a graph representation of a CNF formula, we will quickly glance over what effects the restriction of being a tree would have for the corresponding formulas.

Recall that in primal graphs, the vertices are the variables, and the edges connect variables that appear together in a clause.
In a primal tree, each clause of the corresponding formula can be mapped to an edge in the tree.
As such, each clause contains exactly two variables.
Sibling variables (variables corresponding to sibling nodes) never appear together in a clause.

Dual trees correspond to formulas whose clauses are of unbounded size, and two clauses can share any amount of variables. However, for any three clauses it holds $C_1 \cap C_2 \cap C_3 = \emptyset$, as otherwise there would be a cycle.

An incidence graph that is also a tree has the following property: any two vertices at the same depth are either all variables or all clauses, where depth is the distance from some designated root node.
Every clause has exactly $n+1$ variables, where $n$ is the amount of the corresponding node's children.}

% series-parallel
\textbf{Series-parallel graphs.}
Another useful class of graphs of constant treewidth are {\em series-parallel graphs}. The following definitions are inspired by \cite{DBLP:journals/corr/abs-2004-00547}.

A {\em 2-terminal graph} $(G, s, t)$ is a graph $G$ with two designated nodes $s, t$, often called {\em terminals}.
Two 2-terminal graphs $(G_1, s_1, t_1)$, $(G_2, s_2, t_2)$ can be composed as follows:
series-composition results in a 2-terminal graph $(G, s_1, t_2)$, by a disjoint union of the two graphs and then contracting $t_1, s_2$ into one vertex;
and parallel-composition yields a 2-terminal graph $(G, s, t)$, by a disjoint union of the two graphs and then contracting $s_1, s_2$ and $t_1, t_2$ into one vertex each, named $s$ and $t$ respectively.

A 2-terminal graph $(G, s, t)$ is {\em series-parallel} if it is either a two-vertex graph with a single edge,
or it results from a sequence of series- or parallel-compositions of two 2-terminal graphs.
A series-parallel graph is a graph $G$ such that there exist terminals $s, t$ where $(G, s, t)$ is a series-parallel 2-terminal graph.

The following result applies:

\begin{fact}[\cite{DBLP:journals/corr/abs-2004-00547}]
	%\textbf{Fact 2}
	If $G$ is a series-parallel graph, then \textbf{tw}$(G) = 2$. 
\end{fact}

\noindent
We consider graphs of formulas that are isomorphic to series-parallel graphs, i.e. there exists a bijection $\iota : V(G_F) \rightarrow V(G_{\text{sp}})$, where $G_F$ is some graph representation of the formula, and $G_{\text{sp}}$ is a series-parallel graph.

% cliques
\textbf{Complete graphs.}
A complete graph has the property that for any vertex $v$, there exists an edge to every other vertex $u$ in the graph.
A complete graph with $n$ vertices is denoted $K_n$ and is also often called an $n$-clique.
The following holds:
\begin{fact}[\cite{DBLP:conf/focs/Bonnet0TW20, DBLP:journals/dam/FischerMR08}]
	%\textbf{Fact 3} \cite{DBLP:conf/focs/Bonnet0TW20, DBLP:journals/dam/FischerMR08}.
	If $G$ is an $n$-clique, then $\mathbf{tw}(G) = n - 1$, $\mathbf{tww}(G) = 0$, and $\mathbf{cw}(G) = 2$.
\end{fact}

\noindent
The result for treewidth is obvious, since decomposition is only possible into one bag of size $n$.

{\color{lightgray} Primal cliques represent formulas that consist of one clause, and the formula contains $n$ variables if its primal graph is an $n$-clique.
If the dual graph is a clique, it represents the fact that there is a non-empty set of variables that appears in every clause.
For incidence graphs this case is not of much interest, as any $n$-clique for $n > 2$ has a cycle of odd length, which rules out the clique being bipartite; a 2-clique represents a simple formula of one clause containing one variable.}

% grids
\textbf{Path and grid graphs.}
A {\em $d$-dimensional $n$-grid} is a graph $G$ with $V(G) = \{1, 2, ..., n\}^d$, such that two vertices $(x_1, x_2, ..., x_d)$ and $(y_1, y_2, ..., y_d)$ are connected by an edge if and only if $\sum_{i=1}^d |x_i - y_i| = 1$ \cite{DBLP:conf/focs/Bonnet0TW20}.
Common are 1-dimensional grids (known as path graphs or linear graphs) as well as 2-dimensional grids (usual planar grid graphs).

The following holds for path graphs:
\begin{fact}[\cite{DBLP:conf/focs/Bonnet0TW20}]
	%\textbf{Fact 4} \cite{DBLP:conf/focs/Bonnet0TW20}.
	If $G$ is a path with $n$ vertices, then $\mathbf{tw}(G)=1$, and $\mathbf{tww}(G) = 3$.
\end{fact}

\noindent
{\color{lightgray} A primal path represents a formula where every clause has exactly two variables, and there are as many clauses as there are edges in the path.
Any variable appears in the formula exactly two times in different clauses.
A formula represented by a dual path contains clauses that share a set of variables with the adjacent clauses, and for any three clauses it holds that $C_1 \cap C_2 \cap C_3 = \emptyset$.
An incidence path's vertices alternate between variables and clauses, and the represented class of formulas is the same as in primal paths case.
It is easy to see that a primal path can be transformed into an incidence path by inserting clause vertices, and an incidence path into a primal path by removing said clause vertices.}

Since a path graph is a tree, its treewidth is 1 due to Fact \ref{gw:tree}.
The following result is for 2-dimensional grids:
\begin{fact}[\cite{DBLP:conf/focs/Bonnet0TW20, PhD:Vatshelle}]
	%\textbf{Fact 5} \cite{DBLP:conf/focs/Bonnet0TW20, PhD:Vatshelle}.
	If $G$ is an $n \times n$ grid, then $\mathbf{tw}(G) = n$, $\mathbf{tww}(G) = 6$, $\mathbf{cw}(G) = n + 1$, $\mathbf{rw}(G) = n - 1$, and $ \frac{1}{3}n \leq \mathbf{mimw}(G) \leq \frac{1}{2} (n+1)$.
\end{fact}

\noindent
{\color{lightgray} Primal and dual grids tend to represent formulas with many clauses due to their density.
Incidence graphs can also happen to be grids, if the variables and clauses are arranged in a checkerboard pattern.}

% interval graphs
\textbf{Interval graphs.}
An {\em interval graph} is a graph whose intersection model is that of intervals in $\mathbb{R}$ \cite{DBLP:conf/sat/SaetherTV14}.
That is, for some set of intervals in reals, the graph's vertices represent these intervals, and two vertices are connected by an edge if the corresponding intervals intersect.

\begin{fact}[\cite{PhD:Vatshelle}]
	%\textbf{Fact 6} \cite{PhD:Vatshelle}.
	If $G$ is an interval graph with $n$ vertices, then $\mathbf{tw}(G) = n - 1$, $\mathbf{mimw}(G) = 1$, and $\mathbf{cw}(G)$ and $\mathbf{rw}(G)$ are both in $\Theta(\sqrt{n})$.
\end{fact}

\noindent
An {\em interval ordering} of a CNF formula is a linear ordering $\leq $ on variables and clauses of the formula, such that for any variable $x$ and clause $C$: if $x \in C$, then for all variables $y$ s.t. $x \leq y \leq C$ it follows that $y \in C$; and if $C \leq x$, then for all clauses $S$ s.t. $C \leq S \leq x$ it follows that $x \in S$ \cite{DBLP:conf/sat/SaetherTV14}.
A CNF formula has an interval ordering iff its incidence graph is an interval graph \cite{DBLP:conf/sat/SaetherTV14}.

\begin{fact}[\cite{DBLP:conf/sat/SaetherTV14}]
	%\textbf{Fact 7} \cite{DBLP:conf/sat/SaetherTV14}.
	Let $F$ be a CNF formula with $m$ clauses and incidence graph $G_F$.
	If $G_F$ is an interval graph, then $\mathbf{psw}(F) \leq m$.
	\label{fact:psw-interval}
\end{fact}

\noindent
As an example, the formula $F=\{ \{x_1, x_2\}, \{\neg x_2, x_3\} \}$ has an interval ordering $x_1 \leq x_2 \leq C_1 \leq C_2 \leq x_3$, and its incidence graph is an interval graph.

% d-degenerate
\textbf{{\em d}-degenerate graphs.}
A graph is said to be {\em $d$-degenerate} if there exists a linear ordering $\leq$ of vertices such that for any vertex $v$, the amount of neighbors $u$ s.t. $v \leq u$ is at most $d$ \cite{PhD:Vatshelle}.

\begin{fact}[\cite{PhD:Vatshelle}]
	%\textbf{Fact 8} \cite{PhD:Vatshelle}.
	If $G$ is a $d$-degenerate graph with $n$ vertices, then $\mathbf{tw}(G) \leq n - \alpha$, $\mathbf{cw}(G) \leq \frac{2n}{3} - 2\beta - 1$, $\mathbf{rw}(G) \leq \frac{n}{3} - \beta$ and $\mathbf{mimw}(G) \leq \frac{n}{3} - \beta$, where $\alpha = \frac{n}{d + 3} - 2$ and $\beta = \frac{n}{9d+3} - 1$.
\end{fact}

\textbf{Lower bound on complete bipartite graphs.}
A complete bipartite graph $K_{m,n}$ is a bipartite graph with two sets of vertices $U, V$ with $|U|=m, |V|=n$, such that every vertex from $U$ is connected by an edge to every vertex from $V$ (and vice versa).
%$K_{m,n}$ is $\min \{m,n\}$-degenerate and has treewidth $\mathbf{tw}(G) \geq \min \{m,n\}$.

\begin{fact}[folklore]
		It holds that $\mathbf{tw}(K_{m,n}) \geq \min \{m,n\}$, $\mathbf{cw}(K_{m,n})=2$.
\end{fact}


