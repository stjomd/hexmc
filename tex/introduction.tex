Propositional satisfiability problem (SAT) is the problem of determining whether a propositional formula in conjunctive normal form (CNF) is satisfiable, i.e. if there exists a truth assignment of variables that satisfies the formula.
Such assignments are called models.
Propositional formulas can be thought of as finite sets of clauses, which in turn are finite sets of literals.
A literal is either a propositional variable or its negation.
Given a CNF formula, a model thus satisfies all its clauses, that is, it sets at least one literal in each clause to true.
Propositional model counting, \#SAT, is a related but harder problem of determining the amount of satisfying truth assignments of a CNF formula.

The complexity class P consists of all problems that can be solved in polynomial time; NP is a class of all problems the solutions of which can be verified in polynomial time. SAT is well known to be NP-complete.
The counting counterpart of NP is the class \#P.
\#SAT has been shown to be \#P-complete and remains \#P-hard even when syntactical restrictions on the input CNF formulas are introduced \cite{DBLP:conf/sat/GanianS17}.
However, there has been research on structural restrictions \cite{DBLP:conf/sat/GanianS17, DBLP:conf/sat/CapelliDM14, DBLP:conf/sat/GanianPSSS22, DBLP:conf/sat/SaetherTV14, DBLP:journals/dam/FischerMR08, DBLP:journals/jda/SamerS10, DBLP:journals/fuin/GanianHO13} which lead to efficient algorithms for \#SAT.

Structural restrictions are defined according to a parameter on some graph associated with the problem instance.
In case of \#SAT, such graphs often describe the relationship between the variables and/or clauses of the formula.
A certain parameter $k$, usually a positive integer, is then determined for this graphic representation of a propositional formula. 
If it's then possible to show that for some fixed $k$, there exists an algorithm that can compute \#SAT in time $f(k)n^{O(1)}$, where $f$ is a computable function and $n$ is the size of the formula, we say that \#SAT is {\em fixed-parameter tractable} when parameterized by $k$ \cite{DBLP:conf/sat/GanianS17}.
As one can see, for a fixed $k$ the term $f(k)$ is a constant, basically meaning that the runtime of the parameterized algorithm is polynomial for such $k$, and such algorithm can therefore be assumed efficient (tractable). The class of fixed parameter tractable problems is denoted FPT.
Its non-parameterized counterpart, FP, is the class of functional problems solvable in polynomial time.

This thesis is organized as follows.
In Section 2, we provide the necessary formal definitions, in particular with regards to graphical representations of formulas, and introduce several structural restrictions, among which are {\em treewidth}, {\em clique-width}, {\em disjoint branches}, {\em twin-width}, {\em rank-width}, {\em ps-width}, and others.
